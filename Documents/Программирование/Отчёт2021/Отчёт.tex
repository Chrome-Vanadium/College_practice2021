%% -*- coding: utf-8 -*-
\documentclass[12pt,a4paper]{scrartcl} 
\usepackage[utf8]{inputenc}
\usepackage[english,russian]{babel}
\usepackage{indentfirst}
\usepackage{misccorr}
\usepackage{graphicx}
\usepackage{amsmath}
\begin{document}
		\begin{titlepage}
		\begin{center}
			\large
			МИНИСТЕРСТВО НАУКИ И ВЫСШЕГО ОБРАЗОВАНИЯ РОССИЙСКОЙ ФЕДЕРАЦИИ
			
			Федеральное государственное бюджетное образовательное учреждение высшего образования
			
			\textbf{АДЫГЕЙСКИЙ ГОСУДАРСТВЕННЫЙ УНИВЕРСИТЕТ}
			\vspace{0.25cm}
			
			Инженерно-физический факультет
			
			Кафедра автоматизированных систем обработки информации и управления
			\vfill
			
			\vfill
			
			\textsc{Отчет по практике}\\[5mm]
			
			{\LARGE \textit{Решение системы линейных алгебраических уравнений методом Гаусса-Жордана}}
			\bigskip
			
			2 курс, группа 2УТС
		\end{center}
		\vfill
		
		\newlength{\ML}
		\settowidth{\ML}{«\underline{\hspace{0.7cm}}» \underline{\hspace{2cm}}}
		\hfill\begin{minipage}{0.5\textwidth}
			Выполнил:\\
			\underline{\hspace{\ML}} А.\,А.~Мугу\\
			«\underline{\hspace{0.7cm}}» \underline{\hspace{2cm}} 2021 г.
		\end{minipage}%
		\bigskip
		
		\hfill\begin{minipage}{0.5\textwidth}
			Руководитель:\\
			\underline{\hspace{\ML}} С.\,В.~Теплоухов\\
			«\underline{\hspace{0.7cm}}» \underline{\hspace{2cm}} 2021 г.
		\end{minipage}%
		\vfill
		
		\begin{center}
			Майкоп, 2021 г.
		\end{center}
	\end{titlepage}
\section{Введение}
\label{sec:intro}
Метод Гаусса — Жордана (метод полного исключения неизвестных) — метод, который используется для решения квадратных систем линейных алгебраических уравнений, нахождения обратной матрицы, нахождения координат вектора в заданном базисе или отыскания ранга матрицы. Метод является модификацией метода Гаусса. 
\section{Ход работы}
\label{sec:exp}
\subsection{Код приложения}
\label{sec:exp:code}
\begin{verbatim}
#include <math.h>
#include <iostream>
using namespace std;

//вариант 2
class Gauss
{
float a[50][50];
int n;
public:
void accept()
{
cout << "Enter no. of variables: ";
cin >> n;
for (int i = 0; i < n; i++)
{
for (int j = 0; j < n + 1; j++)
{
if (j == n)
cout << "Constant no." << i + 1 << " = ";
else
cout << "a[" << i + 1 << "][" << j + 1 << "] = ";
cin >> a[i][j];
}
}
}
void display()
{
for (int i = 0; i < n; i++)
{
cout << "\n";
for (int j = 0; j < n + 1; j++)
{
if (j == n)
cout << " ";
cout << a[i][j] << "\t";
}
}
}

void gauss()//converting augmented matrix to row echelon form
{
float temp;//Line 1
for (int i = 0; i < n; i++)
{
for (int j = i + 1; j < n; j++)
{
temp = a[j][i] / a[i][i];//Line 2
for (int k = i; k < n + 1; k++)
{
a[j][k] -= temp * a[i][k];//Line 3
//a[j][k]-=a[j][i]*a[i][k]/a[i][i];//Line 4
}
}
}
}

void EnterJordan()//converting to reduced row echelon form
{
float temp;
for (int i = n - 1; i >= 0; i--)
{

for (int j = i - 1; j >= 0; j--)
{
temp = a[j][i] / a[i][i];
for (int k = n; k >= i; k--)
{
a[j][k] -= temp * a[i][k];
}
}
}

float *x = new float [n];
for (int i = 0; i < n; i++)//making leading coefficients zero
x[i] = 0;
for (int i = 0; i < n; i++)
{
for (int j = 0; j < n + 1; j++)
{
if (x[i] == 0 && j != n)
x[i] = a[i][j];
if (x[i] != 0)
a[i][j] /= x[i];
}
}
delete[]x;
}
void credits()
{
for (int i = 0; i < n; i++)
{
cout << "\nx" << i + 1 << " = " << a[i][n] << endl;
}
}

};

int main()
{
Gauss obj;
obj.accept();
cout << "\n\nAugmented matrix: \n\n\n";
obj.display();
obj.gauss();
cout << "\n\nRow Echelon form: \n\n\n";
obj.display();
obj.EnterJordan();
cout << "\n\nReduced row echelon form:\n\n\n";
obj.display();
cout << "\n\nSolution: \n\n\n";
obj.credits();
return 0;
}
\end{verbatim}
\subsection{Алгоритм}
\label{sec:example}
\begin{enumerate}
	\item Выбирают первый слева столбец матрицы, в котором есть хоть одно отличное от нуля значение.
	\item Если самое верхнее число в этом столбце ноль, то меняют всю первую строку матрицы с другой строкой матрицы, где в этой колонке нет нуля.
	\item Все элементы первой строки делят на верхний элемент выбранного столбца.
	\item Из оставшихся строк вычитают первую строку, умноженную на первый элемент соответствующей строки, с целью получить первым элементом каждой строки (кроме первой) ноль.
	\item Далее проводят такую же процедуру с матрицей, получающейся из исходной матрицы после вычёркивания первой строки и первого столбца.
	\item После повторения этой процедуры n-1 раз получают верхнюю треугольную матрицу
	\item Вычитают из предпоследней строки последнюю строку, умноженную на соответствующий коэффициент, с тем, чтобы в предпоследней строке осталась только 1 на главной диагонали.
	\item Повторяют предыдущий шаг для последующих строк. В итоге получают единичную матрицу и решение на месте свободного вектора (с ним необходимо проводить все те же преобразования).
\end{enumerate}
\newpage
\section{Пример работы программы}
\label{pic:example}
\begin{figure}[h]
	\centering
	\includegraphics[width=0.4\textwidth]{practice_code1.jpg}
	\caption{Пример вычислений}\label{fig:par}
\end{figure}
\begin{thebibliography}{9}
	\bibitem{Chacon-2016}Чакон С. Штрауб Б. Git для профессионального программиста 2016 г.
	\bibitem{Knuth-2003}Кнут Д.Э. Всё про \TeX. \newblock --- Москва: Изд. Вильямс, 2003 г. 550~с.
	\bibitem{Lvovsky-2003}Львовский С.М. Набор и верстка в системе \LaTeX{}. \newblock --- 3-е издание, исправленное и дополненное, 2003 г.
	\bibitem{Voroncov-2005}Воронцов К.В. \LaTeX{} в примерах. 2005 г.
\end{thebibliography}
\end{document}